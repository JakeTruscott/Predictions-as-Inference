% Options for packages loaded elsewhere
\PassOptionsToPackage{unicode}{hyperref}
\PassOptionsToPackage{hyphens}{url}
%
\documentclass[
]{article}
\usepackage{amsmath,amssymb}
\usepackage{iftex}
\ifPDFTeX
  \usepackage[T1]{fontenc}
  \usepackage[utf8]{inputenc}
  \usepackage{textcomp} % provide euro and other symbols
\else % if luatex or xetex
  \usepackage{unicode-math} % this also loads fontspec
  \defaultfontfeatures{Scale=MatchLowercase}
  \defaultfontfeatures[\rmfamily]{Ligatures=TeX,Scale=1}
\fi
\usepackage{lmodern}
\ifPDFTeX\else
  % xetex/luatex font selection
\fi
% Use upquote if available, for straight quotes in verbatim environments
\IfFileExists{upquote.sty}{\usepackage{upquote}}{}
\IfFileExists{microtype.sty}{% use microtype if available
  \usepackage[]{microtype}
  \UseMicrotypeSet[protrusion]{basicmath} % disable protrusion for tt fonts
}{}
\makeatletter
\@ifundefined{KOMAClassName}{% if non-KOMA class
  \IfFileExists{parskip.sty}{%
    \usepackage{parskip}
  }{% else
    \setlength{\parindent}{0pt}
    \setlength{\parskip}{6pt plus 2pt minus 1pt}}
}{% if KOMA class
  \KOMAoptions{parskip=half}}
\makeatother
\usepackage{xcolor}
\usepackage[margin=1in]{geometry}
\usepackage{color}
\usepackage{fancyvrb}
\newcommand{\VerbBar}{|}
\newcommand{\VERB}{\Verb[commandchars=\\\{\}]}
\DefineVerbatimEnvironment{Highlighting}{Verbatim}{commandchars=\\\{\}}
% Add ',fontsize=\small' for more characters per line
\usepackage{framed}
\definecolor{shadecolor}{RGB}{248,248,248}
\newenvironment{Shaded}{\begin{snugshade}}{\end{snugshade}}
\newcommand{\AlertTok}[1]{\textcolor[rgb]{0.94,0.16,0.16}{#1}}
\newcommand{\AnnotationTok}[1]{\textcolor[rgb]{0.56,0.35,0.01}{\textbf{\textit{#1}}}}
\newcommand{\AttributeTok}[1]{\textcolor[rgb]{0.13,0.29,0.53}{#1}}
\newcommand{\BaseNTok}[1]{\textcolor[rgb]{0.00,0.00,0.81}{#1}}
\newcommand{\BuiltInTok}[1]{#1}
\newcommand{\CharTok}[1]{\textcolor[rgb]{0.31,0.60,0.02}{#1}}
\newcommand{\CommentTok}[1]{\textcolor[rgb]{0.56,0.35,0.01}{\textit{#1}}}
\newcommand{\CommentVarTok}[1]{\textcolor[rgb]{0.56,0.35,0.01}{\textbf{\textit{#1}}}}
\newcommand{\ConstantTok}[1]{\textcolor[rgb]{0.56,0.35,0.01}{#1}}
\newcommand{\ControlFlowTok}[1]{\textcolor[rgb]{0.13,0.29,0.53}{\textbf{#1}}}
\newcommand{\DataTypeTok}[1]{\textcolor[rgb]{0.13,0.29,0.53}{#1}}
\newcommand{\DecValTok}[1]{\textcolor[rgb]{0.00,0.00,0.81}{#1}}
\newcommand{\DocumentationTok}[1]{\textcolor[rgb]{0.56,0.35,0.01}{\textbf{\textit{#1}}}}
\newcommand{\ErrorTok}[1]{\textcolor[rgb]{0.64,0.00,0.00}{\textbf{#1}}}
\newcommand{\ExtensionTok}[1]{#1}
\newcommand{\FloatTok}[1]{\textcolor[rgb]{0.00,0.00,0.81}{#1}}
\newcommand{\FunctionTok}[1]{\textcolor[rgb]{0.13,0.29,0.53}{\textbf{#1}}}
\newcommand{\ImportTok}[1]{#1}
\newcommand{\InformationTok}[1]{\textcolor[rgb]{0.56,0.35,0.01}{\textbf{\textit{#1}}}}
\newcommand{\KeywordTok}[1]{\textcolor[rgb]{0.13,0.29,0.53}{\textbf{#1}}}
\newcommand{\NormalTok}[1]{#1}
\newcommand{\OperatorTok}[1]{\textcolor[rgb]{0.81,0.36,0.00}{\textbf{#1}}}
\newcommand{\OtherTok}[1]{\textcolor[rgb]{0.56,0.35,0.01}{#1}}
\newcommand{\PreprocessorTok}[1]{\textcolor[rgb]{0.56,0.35,0.01}{\textit{#1}}}
\newcommand{\RegionMarkerTok}[1]{#1}
\newcommand{\SpecialCharTok}[1]{\textcolor[rgb]{0.81,0.36,0.00}{\textbf{#1}}}
\newcommand{\SpecialStringTok}[1]{\textcolor[rgb]{0.31,0.60,0.02}{#1}}
\newcommand{\StringTok}[1]{\textcolor[rgb]{0.31,0.60,0.02}{#1}}
\newcommand{\VariableTok}[1]{\textcolor[rgb]{0.00,0.00,0.00}{#1}}
\newcommand{\VerbatimStringTok}[1]{\textcolor[rgb]{0.31,0.60,0.02}{#1}}
\newcommand{\WarningTok}[1]{\textcolor[rgb]{0.56,0.35,0.01}{\textbf{\textit{#1}}}}
\usepackage{graphicx}
\makeatletter
\def\maxwidth{\ifdim\Gin@nat@width>\linewidth\linewidth\else\Gin@nat@width\fi}
\def\maxheight{\ifdim\Gin@nat@height>\textheight\textheight\else\Gin@nat@height\fi}
\makeatother
% Scale images if necessary, so that they will not overflow the page
% margins by default, and it is still possible to overwrite the defaults
% using explicit options in \includegraphics[width, height, ...]{}
\setkeys{Gin}{width=\maxwidth,height=\maxheight,keepaspectratio}
% Set default figure placement to htbp
\makeatletter
\def\fps@figure{htbp}
\makeatother
\setlength{\emergencystretch}{3em} % prevent overfull lines
\providecommand{\tightlist}{%
  \setlength{\itemsep}{0pt}\setlength{\parskip}{0pt}}
\setcounter{secnumdepth}{-\maxdimen} % remove section numbering
\ifLuaTeX
  \usepackage{selnolig}  % disable illegal ligatures
\fi
\IfFileExists{bookmark.sty}{\usepackage{bookmark}}{\usepackage{hyperref}}
\IfFileExists{xurl.sty}{\usepackage{xurl}}{} % add URL line breaks if available
\urlstyle{same}
\hypersetup{
  pdftitle={PAI Walkthrough},
  pdfauthor={Jake S. Truscott},
  hidelinks,
  pdfcreator={LaTeX via pandoc}}

\title{PAI Walkthrough}
\author{Jake S. Truscott}
\date{2024-04-23}

\begin{document}
\maketitle

\hypertarget{predictions-as-inference}{%
\subsection{\texorpdfstring{\textbf{Predictions as
Inference}}{Predictions as Inference}}\label{predictions-as-inference}}

\hypertarget{variables}{%
\subsubsection{\texorpdfstring{\textbf{Variables}}{Variables}}\label{variables}}

\begin{itemize}
\tightlist
\item
  \texttt{data} Data object
\item
  \texttt{model} Model selected from \texttt{caret} CRAN (Ex: `parRF',
  `adaboost')
\item
  \texttt{outcome} Dependent variable of interest
\item
  \texttt{predictors} Predictors of interest (defaults to all variables
  in \texttt{data} except \texttt{outcome}.)
\item
  \texttt{interactions} Interaction term(s) declared as a vector
  separated by `,' (Ex: c(`var1*var2', `var3:var4'))
\item
  \texttt{drop\_vars} Specific variables to drop during omission
  procedure (defaults to all predictors and interactions)
\item
  \texttt{cores} Number of computing cores to allocate (defaults to one)
\item
  \texttt{placebo\_iterations} Number of placebo (shuffling) interations
  to run (defaults to 10)
\item
  \texttt{folds} Number of K-Folds for \texttt{trainControl} in
  \texttt{caret} CRAN (defaults to 5)
\item
  \texttt{train\_split} Proportion of data for train/test split
  (Defaults to 0.8 -- Indicating 80/20 Split)
\item
  \texttt{custom\_tc} Custom \texttt{trainControl} parameters (For
  \texttt{caret} models that require unique/specific parameters)
\item
  \texttt{assign\_factors} Automatically assign levels to prescribe
  factor variables (Defaults to 3 Levels)
\item
  \texttt{list\_drop\_vars} If \texttt{TRUE} assigns drop\_vars as a
  group designated by object in global environment (Ex: vars = c(`var1',
  `var2', `var3') and drops them collectively instead of individually.
\item
  \texttt{seed} Random seed generator (Defaults to \texttt{1234})
\end{itemize}

\hypertarget{sample-run}{%
\subsubsection{\texorpdfstring{\textbf{Sample
Run}}{Sample Run}}\label{sample-run}}

\hypertarget{sandbox-data}{%
\paragraph{\texorpdfstring{\textbf{Sandbox
Data}}{Sandbox Data}}\label{sandbox-data}}

\begin{Shaded}
\begin{Highlighting}[]
\NormalTok{sandbox\_data }\OtherTok{\textless{}{-}} \FunctionTok{data.frame}\NormalTok{(}
  \AttributeTok{var1 =} \FunctionTok{sample}\NormalTok{(}\DecValTok{0}\SpecialCharTok{:}\DecValTok{1}\NormalTok{, }\DecValTok{100}\NormalTok{, }\AttributeTok{replace =} \ConstantTok{TRUE}\NormalTok{),}
  \AttributeTok{var2 =} \FunctionTok{sample}\NormalTok{(}\DecValTok{0}\SpecialCharTok{:}\DecValTok{50}\NormalTok{, }\DecValTok{100}\NormalTok{, }\AttributeTok{replace =} \ConstantTok{TRUE}\NormalTok{),}
  \AttributeTok{var3 =} \FunctionTok{c}\NormalTok{(}\FunctionTok{sample}\NormalTok{(}\DecValTok{0}\SpecialCharTok{:}\DecValTok{1}\NormalTok{, }\DecValTok{99}\NormalTok{, }\AttributeTok{replace =} \ConstantTok{TRUE}\NormalTok{), }\DecValTok{2}\NormalTok{),}
  \AttributeTok{var4 =} \FunctionTok{sample}\NormalTok{(}\DecValTok{0}\SpecialCharTok{:}\DecValTok{50}\NormalTok{, }\DecValTok{100}\NormalTok{, }\AttributeTok{replace =} \ConstantTok{TRUE}\NormalTok{),}
  \AttributeTok{var5 =} \FunctionTok{sample}\NormalTok{(}\DecValTok{0}\SpecialCharTok{:}\DecValTok{1}\NormalTok{, }\DecValTok{100}\NormalTok{, }\AttributeTok{replace =} \ConstantTok{TRUE}\NormalTok{),}
  \AttributeTok{var6 =} \FunctionTok{sample}\NormalTok{(}\DecValTok{0}\SpecialCharTok{:}\DecValTok{50}\NormalTok{, }\DecValTok{1000}\NormalTok{, }\AttributeTok{replace =} \ConstantTok{TRUE}\NormalTok{))}
\end{Highlighting}
\end{Shaded}

\emph{Note}: Introduced sparse data problem with \texttt{var3} --
Routine will know to omit variable and any interactions containing the
variable, same with any variable where there is sparse data variance in
train/test split.

\hypertarget{test-run}{%
\paragraph{\texorpdfstring{\textbf{Test
Run}}{Test Run}}\label{test-run}}

\begin{center}\includegraphics[width=6.92in]{output} \end{center}

\begin{Shaded}
\begin{Highlighting}[]
\NormalTok{pai\_test }\OtherTok{\textless{}{-}} \FunctionTok{pai\_function}\NormalTok{(}\AttributeTok{data =}\NormalTok{ sandbox\_data,}
            \AttributeTok{model =} \StringTok{\textquotesingle{}parRF\textquotesingle{}}\NormalTok{,}
            \AttributeTok{outcome =} \StringTok{\textquotesingle{}var1\textquotesingle{}}\NormalTok{,}
            \AttributeTok{predictors =} \ConstantTok{NULL}\NormalTok{,}
            \AttributeTok{interactions =} \FunctionTok{c}\NormalTok{(}\StringTok{\textquotesingle{}var4*var5\textquotesingle{}}\NormalTok{),}
            \AttributeTok{cores =} \DecValTok{5}\NormalTok{) }\CommentTok{\#Test PAI Run}
\end{Highlighting}
\end{Shaded}

\hypertarget{diagnostic-output}{%
\paragraph{\texorpdfstring{\textbf{Diagnostic
Output}}{Diagnostic Output}}\label{diagnostic-output}}

\hypertarget{placebo-iterations}{%
\subparagraph{\texorpdfstring{\textbf{Placebo
Iterations}}{Placebo Iterations}}\label{placebo-iterations}}

\includegraphics{pai_walkthrough_files/figure-latex/unnamed-chunk-5-1.pdf}

\hypertarget{push-protocol}{%
\subparagraph{\texorpdfstring{\textbf{Push
Protocol}}{Push Protocol}}\label{push-protocol}}

\includegraphics{pai_walkthrough_files/figure-latex/unnamed-chunk-6-1.pdf}

\begin{verbatim}
## TableGrob (2 x 1) "arrange": 2 grobs
##      z     cells    name            grob
## var2 1 (1-1,1-1) arrange gtable[arrange]
## var4 2 (2-2,1-1) arrange gtable[arrange]
\end{verbatim}

\hypertarget{bootstrap-confidence-intervals}{%
\subparagraph{\texorpdfstring{\textbf{Bootstrap Confidence
Intervals}}{Bootstrap Confidence Intervals}}\label{bootstrap-confidence-intervals}}

\textbf{Var-Level Confidence Intervals}

\textbf{Density of Boostrap Accuracies}

\end{document}
